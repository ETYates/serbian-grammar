\chapter{THE ALPHABET}
        The Cyrillic alphabet as used in Serbia consists of thirty letters. It
        originally contained more, but was reformed and simplified in the first
        half of the nineteenth century by the great Serbian philologist and author
        Vuk Stefanović Karadžić (1787--1864), who, by this means, brought it to
        complete accord with the phonetics of the modern spoken language. Being
        originally founded on the Greek, the order of the letters is mainly that of
        the Greek alphabet. The Croatian alphabet naturally follows the order of
        the Latin, but in the accompanying scheme (pages \textbf{12} and
        \textbf{13}) this order has been altered in order to show the correspondence
        between it and the Serbian Cyrillic.

        The following is the Croatian alphabet in the order of the Latin
        letters, with the Cyrillic equivalents :

        \begin{tabularx}{\textwidth}{X X X X}
            {a A} & а А & {l L} & {л Л}\\
            {b B} & б Б & {lj Lj} & {љ Љ}\\
            {c C} & {ц Ц} & {m M} & {м М}\\
            {č Č} & {ч Ч} & {n N} & {н Н}\\
            {ć Ć} & {ћ Ћ} & {n Nj} & {њ Њ}\\
            {d D} & {д Д} & {o O} & {о О}\\
            {dž ǵ Dž} & {џ Џ} & {p P} & {п П}\\
            {đ gj\foot{Also, less commonly, dj, Dj.} Đ Gj} & {ђ Ђ} & {r R} & {р Р}\\
            {e E} & {е Е} & {s S} & {с С}\\
            {f F} & {ф Ф} & {š Š} & {ш Ш}\\
            {g G} & {г Г} & {t T} & {т Т}\\
            {h H} & {х Х} & {u U} & {у У}\\
            {i I} & {и И} & {v V} & {в В}\\
            {i J} & {ј Ј} & {z Z} & {з З}\\
            {k K} & {к К} & {ž Ž} & {ж Ж}\\
        \end{tabularx}

        It will be seen that in several cases the Croatian Latin alphabet
        emlpoys double letters or letters with diacritic signs over them where
        the Serbian Cyrillic constantly employs only one letter.
        It even has alternative signs to represent certain sounds, the reason
        being that uniformity of spelling in Croatia has not yet been achieved,
        while in Serbia it has, e.g. Serbian ђ can be represented
        in Croatian by gj or đ or dj, of which the first two are the most
        usual. Of the other alternative signs, lj is commoner than l, nj than
        ń, dž than ǵ.
        
        The following is the Serbian alphabet in the order of the Cyrillic
        letters, with the Latin (Croatian) equivalents, and the cursive script
        in both alphabets :
        \newpage
        
        \hspace*{-1.75cm}\begin{tabularx}{\paperwidth}{c|c|c|c|c|c}
            {} & \multicolumn{2}{c|}{\textbf{\Large CYRILLIC}} & \multicolumn{2}{c|}{\textbf{\Large LATIN}} &\\
            \hline
            \shortstack{Ord.\\ No.} & Printed & Written & Printed & Written & Pronunciation\\
            \hline
            1 & {\LARGE а А} &  & a A & & English \emph{a} as in \emph{father}.\\
            2 & {\LARGE б Б} &  & {\LARGE b B}    & & \\
            3 & {\LARGE в В} &  &  {\LARGE v V}   & & \\
            4 & {\LARGE г Г} &  &  {\LARGE g G}   & & \\
            5 & {\LARGE д Д} &  & {\LARGE d D}    & & \\
            6 & {\LARGE ђ Ђ} &  & \shortstack{\LARGE đ (dj), Đ\\\LARGE gj \hfill Dj}    & & \\
            7 & {\LARGE е Е} &  & {\LARGE }    & & \\
            8 & {\LARGE ж Ж} &  & {\LARGE }    & & \\
            9 & {\LARGE з З} &  & {\LARGE }    & & \\
            10 & {\LARGE и И} &  & {\LARGE }    & & \\
            11 & {\LARGE ј Ј} &  & {\LARGE }    & & \\
            12 & {\LARGE к К} &  & {\LARGE }    & & \\
            13 & {\LARGE л Л} &  & {\LARGE }    & & \\
            14 & {\LARGE љ Љ} &  & {\LARGE }    & & \\
            15 & {\LARGE м М} &  & {\LARGE }    & & \\
            16 & {\LARGE н Н} &  & {\LARGE }    & & \\
            17 & {\LARGE њ Њ} &  & {\LARGE }    & & \\
        \end{tabularx}
        
        \newpage
        \makebox[\textwidth]{\large \textsc{Note on Foreign Words}} 

        Foreign proper names when transliterated in Cyrillic are spelt 'phonetically', e.g.

        \textit{Shakespeare} = Ш\`{е}кспир ; \textit{Glasgow} =
        {Гл\`{а}згоу} or {Гл\={а}згов} ; \textit{William} =
        {В\`{и}љем} ; \textit{John} = Џ\~{о}н.

        Foreign words as a rule have to conform to the Serbo-Croatian rules of phonetics and orthography, e.g. 
        
        \makebox[\textwidth]{\textit{professor} = {про̏фесор} ; \textit{engineer} = {инж\`{и}н\={е}р}}
