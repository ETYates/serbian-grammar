\chapter{THE SERBIAN LANGUAGE}
        \indent T{\scriptsize HE} Serbian language is one of the Slavonic languages,\foot{The
        Slavonic languages fall into three groups, the Eastern (Russian, i.e. Great
        Russian and Little Russian), the Southern (Bulgarian, Serbo-Croation, and
        Slovene), and the Western (Bohemian or Chekh or Czech, Slovak, Polish, and
        Lusatian-Wendish or Sorbish).} and therefore also one of the family of
        Indo-European languages. It is spoken by more than ten millions of Serbs and
        Croats living in the following countries and territories : the kingdoms of
        Serbia and Montenegro, Bosnia and Hercegovina, Dalmatia and the islands,
        Croatia and Slavonia, in parts of Istria and in the former 'Serbian Duchy'
        (\textit{Srspka Voivodina}) in Southern Hungary, which includes the districts
        of purely Serbian nationality, known as \textit{Baranja}, \textit{Banat}, and
        \textit{Bačka}. There are also large colonies of Serbs and Croats in the United
        States and in South America. 

        The language of both Serbs and Croats is, with the exception of inevitable
        differences of dialect and vocabulary, one and the same ; thus it is customary
        to speak of it as the \textit{Serbo-Croatian language}. Very closely allies to
        the Serbo-Croatian language, of which it ,ay be considered almost a dialect, is
        the language of the one and a half million Slovenes who inhabit the southern
        parts of the provinces of Styria and Carinthia, the province of Carinola, and
        the districts of Trieste and Gorica (Gorizia) in Austria. The Serbs, Croats,
        and Slovenes are all included in the term \textit{Southern Slavs} or
        \textit{Jugo-Slavs} (\textit{jug}, pronounce \textit{yug=south} in Serbian).

        The Serbs, being members of the Eastern or Orthodox Church, use the alphabet
        known as the Cyrillic, the Croats and Slovenes, being Roman Catholics, use the
        Latin alphabet. The Cyrillic alphabet is also used in Russia and Bulgaria, i.e.
        by all orthodox Slavs. The Latin alphabet, as used by the Croats and Slovenes,
        is the same as that used in England except for the fact that a few consonants
        have been furnished with diacritic signs to represent certain complex sounds.

        The Cyrillic alphabet is so called after St. Cyril, who, with his brother
        Methodius, converted the Slavs in Moravia in the ninth century, and are known
        as the 'Slavonic Apostles'. The Slavs of the Balkans were actually converted by
        their disciples. These two missionaries were Greeks of Salonica, but
        they knew the language of the Balkan Slavs, who at that time were
        already settled up to a few miles of Salonica, and St. Cyril
        is credited with the invention of this alphabet to help the success of his
        mission, and to enable the Holy Scriptures to be written in the various
        Slavonic languages. This alphabet is founded on the Greek, but contains a
        number of letters representing sounds which did not exist in Greek. Some of
        these letters are supposed to have been borrowed from Semitic sources, others
        were freshly elaborated.

        The foreigner should learn the Cyrillic alphabet, but he must also
        sooner or later make himself familiar with the language as expressed by
        the Latin alphabet. A knowledge of both alphabets is essential both
        from the literary and from the practical points of view.
        
        It may be pointed out that the Cyrillic alphabet as used in Serbia and
        Montenegro, \&c., is purely phonetic in that each signle sign by itself
        represents one and only one sound in the language, which can hardly be said of
        any other European alphabet. Conversely, there are no sounds in the language
        other than those expressed by its alphabet. The same hold good of Croatian,
        except that one or two double letters are still used.
