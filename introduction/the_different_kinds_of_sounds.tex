
\vspace*{16pt}
\chapter{THE DIFFERENT KINDS OF SOUNDS}
\section{Vowels}

Besides the five normal vowels--а, е, и, о, у--р can also rank as a vowel when
it is (1) between to consonants, or (2) at the beginning of a word before a
consonant; in these cases it is strongly rolled as in Scotland, e.g.

1. С\textdoublegrave{р}бин=\emph{a Serb} (\emph{masc.}) ;
т\`{р}говац=\emph{merchant} ; чв\~{р}ст=\emph{firm}.\foot{Even in words of
foreign origin, e.g. трпез\`{а}рија=\emph{dining-room}, from the Greek
\textgreek{τραπεζάριον}.}

2. \textdoublegrave{р}ђа=\emph{rust} ; \`{р}вати се=\emph{to wrestle} ;
\textdoublegrave{р}ђав=\emph{bad}.

р  very seldom occurs as a vowel-sound before or after a vowel; when it does it
is  indicated by two dots, e.g.
\newpage

г\"{р}\'{о}це (3 syllables) = \emph{throat} (diminutive).

за\`{\"{р}}ђати (4 syllables) = \emph{to become rusty}.

All vowels, including р, may be either short or long.

\smallskip
{\hfill \large \textbf{`Movable A'} \hfill}

Particular mention must be made of what is knows as the `movable a'. In Serbian
only the following four groups of consonants are possible at the end of words :
ст, шт, зд, жд; when a words would end in any other group than these, an a is
inserted in the \emph{nom, sing.}, but disappears in the other cases where the
word naturally ends in a vowel; but in the \emph{gen. plur.} the a reappears in
these words, a phenomenon caused by the fact that the invariable long
\emph{final} a of this case is of comparatively modern origin.  E.g.

\emph{Nom. sing.} к\`{о}нац=\emph{cotton, cotton, thread}

\emph{Gen. sing.} к\'{о}нца.

\emph{Gen. plur.} к\`{о}н\={а}ц\={а}

It is very frequent in the \emph{nom, sing. masc.} of adjectives, e.g.
ж\'{е}дан (\emph{masc})=\emph{thirsty}, but ж\'{е}дна (\textit{fem.}).

In the case of foreign words practice varies ; thus one finds both
ф\textdoublegrave{а}кт and ф\textdoublegrave{а}кат. 

\smallskip
{\hfill \large \textbf{Final л and о} \hfill}

Final л of a syllable, and especially of a word, very frequently becomes о.
In words which originally ended in -ол in the \emph{nom. sing.} the two о's
then combine into one long vowel, but the л reappears in the othercases, e.g.

в\~{о} (\emph{m.})=\emph{ox}, gen. sing. в\'{о}ла. 

ст\~{о} (\emph{m.})=\emph{Table}, gen.sing. ст\`{о}ла.

с\~{о}=\emph{salt}, gen.sing. с\`{о}ли, the nom. sing. of which was originally
вол, стол, and сол.

In other cases the л appears as о after another vowel when
final, reappearing in other cases, e.g.

б\textdoublegrave{е}о=\emph{white} (nom. sing, masc), but б\'{е}ла
=\emph{white} (nom. sing, fem.), б\~{е}ли=\emph{white} (nom. plur. masc).
This phenomenon occurs most frequently in the past participle of the verbs, e.g.

\`{и}мао=\emph{(he) had} (masc. sing.), but \`{и}мала=\emph{(she) had} (fem. sing.). 

\pagebreak

It may also occur in the middle of words when n. is at the end
of a syllable, e.g.

се\`{о}ба=\emph{migration} (originally селба).

вл\'{а}даоца, gen. sing. of вл\`{а}далац=\emph{ruler} (e,g, \emph{king}).

Cf. also Бе\`{о}град=\emph{Belgrade} (lit. \emph{the white city}, originally Бел-град).

\section{Consonants}

The consonants, according to the manner of their
articulation, fall into the two groups :

1. Voiced : б, в, г, д, ђ, ж, з, џ.

2. Voiceless : п, ф, к, т, ћ, ш, с, ч, ц, х.

\smallskip
{\hfill \textit{Rule of the assimilation of Consonants} \hfill}

When a voiced and a voiceless consonant come together,\linebreak
assimilation takes place, i.e. both must be either voiced \linebreak 
or voiceless : (1) a voiceless consonant becomes voiced before
a voiced consonant, and (2) vice versa, e.g.

\begin{itemize}
    \item[(1)] св\textdoublegrave{а}дба (\emph{f.})=\emph{wedding} is derived
        from сват+ба (св\textdoublegrave{а}т (\emph{m}.)\\
        \hspace*{3in}=\emph{wedding guest})\\
        \`{о}таџбина ({\it f.}) = \textit{fatherland}\quad ,,\qquad,, \quad
        отач+бина (\`{о}тац ({\it m.})\\
        \hspace*{3in}=\emph{father})  
    \item[(2)] с\textdoublegrave{р}пски ({\it adj.}) =
        \emph{Serbian}\qquad,,\qquad,, \quad 
        срб+ски (с\textdoublegrave{р}бин ({\it m.}))\\
        \hspace*{3in}=\emph{Serbian} ({\it m.})\\
        вр\'{а}пци ({\it nom. pl.}) =\emph{sparrows}\qquad,, \quad
        враб + ци\quad(вр\'{а}бац\\
        \hspace*{3in}({\it m.})= \emph{sparrow})
\end{itemize}

Exceptions: д remains before с and ш, e.g.

\qquad\qquad пр\'{е}дседник ({\it m.}) = \emph{president.}

\qquad\qquad одшкринути = \emph{to open slightly.}

в never changes into ф and does not change preceding voiceless consonants, e.g.

\qquad\qquad к\`{о}л\={е}вка ({\it f.}) (not колефка) = \textit{cradle}.

\qquad\qquad кл\'{е}тва ({\it f.}) (not кледва) = \textit{curse}.

\grammarnote{Most Important Phonetic Rules}

I. The gutterals к, г, х are `softened' when followed (1) by е and (2) by и, as follows :
\newpage

1. (a) Before е, к changes into ч, г into ж, х into ш, in voc. sing. of masculine nouns, e.g.

Nom. sing., v\~{у}к ({\it m}), \emph{wolf}, voc. sing. в\~{у}че.

\quad,, \qquad ,, \qquad б\~{о}г (\it ) 
