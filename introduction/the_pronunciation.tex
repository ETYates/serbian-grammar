\chapter{THE PRONUNCIATION}

        The pronunciation of Serbo-Croatian is infinitely easier for
        English-speaking people than is that of any of the other Slavonic
        languages. The rule in Serbo-Croatian is 'to write as you speak and to
        speak as you write' (Vuk, cf. p. 10). The pronunciation of each
        individual letter is in all cases the same, therefore the only
        difficulty is to learn the value of each letter.

        The vowels {и, е, а, о, у} are all pronounced 'openly' as
        in Italian, cf. p. 12f.

        The great majority of the consonants also present no difficulty
        whatever. The only consonants which call for special remark are the
        following: ш and ж, ч
        and џ, ћ and ђ.

        ш is a \textit{voiceless}\foot{The difference between a
        \textit{voiceless} and a \textit{voiced} consonant is that a
        \textit{voiceless} consonant is pronounced with breath from the mouth
        \textit{only}, while to pronounce a \textit{voiced} consonant a stream of
        breath from the chest is necessary.} consonant exactly like English \textit{sh}
        ; ж is the corresponding \textit{voiced} consonant pronounced
        like \textit{s} in the English word \textit{pleasure}, or like \textit{j} in
        the French word \textit{jour}.

        ч is a \textit{voiceless} consonant exactly like English
        \textit{ch} in \textit{chalk} ; џ is the corresponding
        \textit{voiced} consonant pronounced like \textit{j} in the English
        word \textit{John}.

        The only difficulty is with the two consonants ћ and
        ђ, though it is by no means insurmountable. To pronounce
        these two consonants the teeth must be brought close together and the
        lips slightly opened. The blade\foot{The blade is the part of the
        tongue immediately behind the point and including it.} of the tongue
        must cleave to the inside of the gums of the upper teeth and be
        slightly drawn back at the moment when the stream of breath comes out
        of the chest throught the mouth. The important point is that
        ћ is a \textit{voiceless} and ђ the corresponding \textit{voiced}
        consonant. Thus ћ and ђ correspond to ч and џ and are very similar to
        them in sound, only they are palatal consonants,\foot{ћ and ђ, besides
        being the result respectively of т+ј and д+ј (cf. p. 18), are also the
        result, in words of comparatively modern formation, of к+ј and к+е, г+ј,
        г+е, e.g. ћ\`{о}шак = \emph{corner} (from Turkish \emph{kiushk}, cf.
        \emph{kiosque, a pavilion}), Маћ\`{е}д\={о}нија=\emph{Macedonia} (к+е),
        М\`{а}ђ\={а}р (also M\`{а}џ\={а}р) = \emph{Magyar}, Ђ\~{о}рђе = \emph{George},
        ђен\`{е}р\={а}л = \emph{general}, though there is now no к or г audible in
        these words.} which ч and џ are not.

        The consonant {х} before a consonant, as in
        хв\'{а}ла = \textit{thanks}, is pronounced like \textit{ch}
        in Scottish \textit{loch}, but before a vowel like an ordinary English
        \textit{h}, as in х\`{а}ртија = \textit{paper}.

        It is important to notice the difference between л and љ, and between н
        and њ ; љ and њ are the \textit{softened} or \textit{palatal} forms of
        л and н, just as ћ and ђ are the \textit{softened} or \textit{palatal}
        forms of т and д. Their pronunciation is perfectly easy and natural for
        English-speaking people except at the end of words, a position in which
        for that matter these letters in Serbo-Croatian seldom occur; in the
        middle of words they sound like \textit{l} and \emph{n} in the English
        words \emph{million} and \emph{new}. 
