\documentclass[11pt]{book}
\usepackage{fontspec}
\usepackage[center,md]{titlesec}
\usepackage{setspace}
\usepackage{indentfirst}
\usepackage{amsmath}
\usepackage{calligra}
\usepackage[T2A]{fontenc}
\usepackage{afterpage}
\usepackage{tabularx}
\usepackage{enumitem}
\setlist{nosep}
\setlist[itemize]{leftmargin=0.7cm}
\usepackage[safe]{tipa}
\usepackage{polyglossia}
\usepackage{fancyhdr}
\usepackage{geometry}
\geometry{
    left=0.475in,
    right=0.475in,
    top=0.75in,
    bottom=0.75in,
    papersize={5.44in, 8.50in}
    }
\setlength{\headsep}{0.1in}
\usepackage{graphicx}
\setdefaultlanguage{english}
\setmainfont[Ligatures=TeX]{Old Standard}
\setotherlanguage{serbian}
\setotherlanguage[variant=polytonic]{greek}
\newfontfamily\serbianfont[Ligatures=TeX]{Old Standard}
\newfontfamily\euroscript{EuroScript.ttf}
\newfontfamily{\greekfont}[Ligatures=TeX]{GFS Porson}
\titleclass{\part}{top}
\let\Chaptermark\chaptermark
\def\chaptermark#1{\def\Chaptername{#1}\Chaptermark{#1}}
\titleformat{\part}[display]
  {\normalfont\huge}{\centering\partname\ \thepart}{20pt}{\Large\centering}
\titlespacing*{\part}{0pt}{30pt}{5pt}
\titleclass{\chapter}{straight}
\titleformat{\chapter}{}{\centering\normalfont \thechapter.}{5pt}{\centering}
\renewcommand{\thesection}{\arabic{section}}
\titleformat{\section}{\large}{\centering\textbf{\thesection.}}{5pt}{\bf \large\centering}
\titlespacing*{\section}{5pt}{2pt}{5pt}
\titlespacing*{\chapter} {0pt}{0pt}{2pt}
% \titleformat{\chapter}{\normalfont\LARGE\bfseries}{\thechapter}{}{}
\newcommand{\foot}[1]{\setstretch{1}\footnote{\normalsize #1}}
\renewcommand*\footnoterule{}
\pagestyle{fancy}
\fancyhf{} % sets both header and footer to nothing
\renewcommand{\headrulewidth}{0pt}
\renewcommand{\chaptermark}[1]{\fancyhead[CO]{\large #1} \fancyhead[LE,RO]{\thepage}}
\fancyfoot[CE,CO]{}
\renewcommand{\partmark}[1]{\markleft{#1}}
\newcommand{\grammarnote}[1]{{\phantom{} \hfill \textsc{#1} \hfill}}
\newcommand\eng{English }
\newcommand\fra{French }
\newcommand\masc{\text{({\it m.})}}
\newcommand\femn{({\it f.})} 
\newcommand\neut{({\it n.})}
\newcommand{\tdg}[1]{\textdoublegrave{#1}}
\newcommand{\trc}[1]{\textroundcap{#1}}
\newcommand{\adj}{({\it adj.})}

\begin{document}
\begin{titlepage}
    \begin{center}
        \vspace*{\fill}
        {\large OXFORD UNIVERSITY PRESS}\\
        \textsc{london\qquad edinburgh \qquad glasgow \qquad new york}\\
        \textsc{toronto \quad melbourne \quad cape town \quad bombay}\\[4pt]
        {\large HUMPHREY MILFORD}\\
        \textsc{\footnotesize publisher to the university}
        \vspace*{\fill}
        \newpage
        {\Huge SERBIAN GRAMMAR}\\
        \vspace{0.5cm}
        BY\\
        \vspace{0.5cm}
        {\Large DRAGUTIN SUBOTIĆ}\\[5pt]
        \textsc{Ph.D., Munich}\\
        \vspace{0.5cm}
        AND\\
        \vspace{0.5cm}
        {\Large NEVILL FORBES, M.A.}\\[5pt]
        {\scriptsize READER IN RUSSIAN AND THE OTHER SLAVONIC LANGUAGES}\\
        {\scriptsize IN THE UNIVERSITY OF OXFORD}\\
        \vfill
        {\large OXFORD}\\
        {\large AT THE CLARENDON PRESS}\\
        {\large 1918}
    \end{center}
\end{titlepage}
\null\newpage

\part*{\LARGE PREFACE}

{\onehalfspacing \large \textsc{The} title of this book has been chosen for the sake of simplicity. The
full name of the language is Serbo-Croatian. It must be emphasized that
Croatian, except for slight differences of of dialiect and vocabulary, is
absolutely the same language as Serbian, only written in the Latin alphabet
with diacritic signs. Knowledge of both the Cyrillic and Latin {Croatian}
alphabets is indispensable to any student of Serbo-Croatian, therefore it is
recommended to practise as much as possible the transcription of words written
in Cyrillic into Latin, and vice versa.

In the \textit{English} excersizes the sentences have sometimes been framed
acording to the rules of \textit{Serbian} syntax, in order to accustom the
student to its peculiarities.

We wish to thank Mr. \v{S}uvakovi\'{c} for the time and labour he\\[1.5pt]  
has given us by helping with the accentuation.}

\hfill D. S.\hspace{1cm}

\hfill N. F.\hspace{1cm}

\part*{INTRODUCTION}
\fancyhead[CE]{\large INTRODUCTION}
\input{introduction/intro.tex}

\chapter{THE SERBIAN LANGUAGE}
        \textsc{The} Serbian language is one of the Slavonic languages,\foot{The
        Slavonic languages fall into three groups, the Eastern (Russian, i.e. Great
        Russian and Little Russian), the Southern (Bulgarian, Serbo-Croation, and
        Slovene), and the Western (Bohemian or Chekh or Czech, Slovak, Polish, and
        Lusatian-Wendish or Sorbish).} and therefore also one of the family of
        Indo-European languages. It is spoken by more than ten millions of Serbs and
        Croats living in the following countries and territories : the kingdoms of
        Serbia and Montenegro, Bosnia and Hercegovina, Dalmatia and the islands,
        Croatia and Slavonia, in parts of Istria and in the former 'Serbian Duchy'
        (\textit{Srspka Voivodina}) in Southern Hungary, which includes the districts
        of purely Serbian nationality, known as \textit{Baranja}, \textit{Banat}, and
        \textit{Bačka}. There are also large colonies of Serbs and Croats in the United
        States and in South America. 

        The language of both Serbs and Croats is, with the exception of inevitable
        differences of dialect and vocabulary, one and the same ; thus it is customary
        to speak of it as the \textit{Serbo-Croatian language}. Very closely allies to
        the Serbo-Croatian language, of which it ,ay be considered almost a dialect, is
        the language of the one and a half million Slovenes who inhabit the southern
        parts of the provinces of Styria and Carinthia, the province of Carinola, and
        the districts of Trieste and Gorica (Gorizia) in Austria. The Serbs, Croats,
        and Slovenes are all included in the term \textit{Southern Slavs} or
        \textit{Jugo-Slavs} (\textit{jug}, pronounce \textit{yug=south} in Serbian).

        The Serbs, being members of the Eastern or Orthodox Church, use the alphabet
        known as the Cyrillic, the Croats and Slovenes, being Roman Catholics, use the
        Latin alphabet. The Cyrillic alphabet is also used in Russia and Bulgaria, i.e.
        by all orthodox Slavs. The Latin alphabet, as used by the Croats and Slovenes,
        is the same as that used in England except for the fact that a few consonants
        have been furnished with diacritic signs to represent certain complex sounds.

        The Cyrillic alphabet is so called after St. Cyril, who, with his brother
        Methodius, converted the Slavs in Moravia in the ninth century, and are known
        as the 'Slavonic Apostles'. The Slavs of the Balkans were actually converted by
        their disciples. These two missionaries were Greeks of Salonica, but
        they knew the language of the Balkan Slavs, who at that time were
        already settled up to a few miles of Salonica, and St. Cyril
        is credited with the invention of this alphabet to help the success of his
        mission, and to enable the Holy Scriptures to be written in the various
        Slavonic languages. This alphabet is founded on the Greek, but contains a
        number of letters representing sounds which did not exist in Greek. Some of
        these letters are supposed to have been borrowed from Semitic sources, others
        were freshly elaborated.

        The foreigner should learn the Cyrillic alphabet, but he must also
        sooner or later make himself familiar with the language as expressed by
        the Latin alphabet. A knowledge of both alphabets is essential both
        from the literary and from the practical points of view.
        
        It may be pointed out that the Cyrillic alphabet as used in Serbia and
        Montenegro, \&c., is purely phonetic in that each signle sign by itself
        represents one and only one sound in the language, which can hardly be said of
        any other European alphabet. Conversely, there are no sounds in the language
        other than those expressed by its alphabet. The same hold good of Croatian,
        except that one or two double letters are still used.


\chapter{THE ALPHABET}
        The Cyrillic alphabet as used in Serbia consists of thirty letters. It
        originally contained more, but was reformed and simplified in the first
        half of the nineteenth century by the great Serbian philologist and author
        Vuk Stefanović Karadžić (1787--1864), who, by this means, brought it to
        complete accord with the phonetics of the modern spoken language. Being
        originally founded on the Greek, the order of the letters is mainly that of
        the Greek alphabet. The Croatian alphabet naturally follows the order of
        the Latin, but in the accompanying scheme (pages \textbf{12} and
        \textbf{13}) this order has been altered in order to show the correspondence
        between it and the Serbian Cyrillic.

        The following is the Croatian alphabet in the order of the Latin
        letters, with the Cyrillic equivalents :

        \begin{tabularx}{\textwidth}{X X X X}
            {a A} & а А & {l L} & {л Л}\\
            {b B} & б Б & {lj Lj} & {љ Љ}\\
            {c C} & {ц Ц} & {m M} & {м М}\\
            {č Č} & {ч Ч} & {n N} & {н Н}\\
            {ć Ć} & {ћ Ћ} & {n Nj} & {њ Њ}\\
            {d D} & {д Д} & {o O} & {о О}\\
            {dž ǵ Dž} & {џ Џ} & {p P} & {п П}\\
            {đ gj\foot{Also, less commonly, dj, Dj.} Đ Gj} & {ђ Ђ} & {r R} & {р Р}\\
            {e E} & {е Е} & {s S} & {с С}\\
            {f F} & {ф Ф} & {š Š} & {ш Ш}\\
            {g G} & {г Г} & {t T} & {т Т}\\
            {h H} & {х Х} & {u U} & {у У}\\
            {i I} & {и И} & {v V} & {в В}\\
            {i J} & {ј Ј} & {z Z} & {з З}\\
            {k K} & {к К} & {ž Ž} & {ж Ж}\\
        \end{tabularx}

        It will be seen that in several cases the Croatian Latin alphabet
        emlpoys double letters or letters with diacritic signs over them where
        the Serbian Cyrillic constantly employs only one letter.
        It even has alternative signs to represent certain sounds, the reason
        being that uniformity of spelling in Croatia has not yet been achieved,
        while in Serbia it has, e.g. Serbian ђ can be represented
        in Croatian by gj or đ or dj, of which the first two are the most
        usual. Of the other alternative signs, lj is commoner than l, nj than
        ń, dž than ǵ.
        
        The following is the Serbian alphabet in the order of the Cyrillic
        letters, with the Latin (Croatian) equivalents, and the cursive script
        in both alphabets :
        \newpage
        
        \hspace*{-1.75cm}\begin{tabularx}{\paperwidth}{c|c|c|c|c|c}
            {} & \multicolumn{2}{c|}{\textbf{\Large CYRILLIC}} & \multicolumn{2}{c|}{\textbf{\Large LATIN}} &\\
            \hline
            \shortstack{Ord.\\ No.} & Printed & Written & Printed & Written & Pronunciation\\
            \hline
            1 & {\LARGE а А} &  & a A & & English \emph{a} as in \emph{father}.\\
            2 & {\LARGE б Б} &  & {\LARGE b B}    & & \\
            3 & {\LARGE в В} &  &  {\LARGE v V}   & & \\
            4 & {\LARGE г Г} &  &  {\LARGE g G}   & & \\
            5 & {\LARGE д Д} &  & {\LARGE d D}    & & \\
            6 & {\LARGE ђ Ђ} &  & \shortstack{\LARGE đ (dj), Đ\\\LARGE gj \hfill Dj}    & & \\
            7 & {\LARGE е Е} &  & {\LARGE }    & & \\
            8 & {\LARGE ж Ж} &  & {\LARGE }    & & \\
            9 & {\LARGE з З} &  & {\LARGE }    & & \\
            10 & {\LARGE и И} &  & {\LARGE }    & & \\
            11 & {\LARGE ј Ј} &  & {\LARGE }    & & \\
            12 & {\LARGE к К} &  & {\LARGE }    & & \\
            13 & {\LARGE л Л} &  & {\LARGE }    & & \\
            14 & {\LARGE љ Љ} &  & {\LARGE }    & & \\
            15 & {\LARGE м М} &  & {\LARGE }    & & \\
            16 & {\LARGE н Н} &  & {\LARGE }    & & \\
            17 & {\LARGE њ Њ} &  & {\LARGE }    & & \\
        \end{tabularx}
        
        \newpage
        \makebox[\textwidth]{\large \textsc{Note on Foreign Words}} 

        Foreign proper names when transliterated in Cyrillic are spelt 'phonetically', e.g.

        \textit{Shakespeare} = Ш\`{е}кспир ; \textit{Glasgow} =
        {Гл\`{а}згоу} or {Гл\={а}згов} ; \textit{William} =
        {В\`{и}љем} ; \textit{John} = Џ\~{о}н.

        Foreign words as a rule have to conform to the Serbo-Croatian rules of phonetics and orthography, e.g. 
        
        \makebox[\textwidth]{\textit{professor} = {про̏фесор} ; \textit{engineer} = {инж\`{и}н\={е}р}}


\chapter{THE PRONUNCIATION}

        The pronunciation of Serbo-Croatian is infinitely easier for
        English-speaking people than is that of any of the other Slavonic
        languages. The rule in Serbo-Croatian is 'to write as you speak and to
        speak as you write' (Vuk, cf. p. 10). The pronunciation of each
        individual letter is in all cases the same, therefore the only
        difficulty is to learn the value of each letter.

        The vowels {и, е, а, о, у} are all pronounced 'openly' as
        in Italian, cf. p. 12f.

        The great majority of the consonants also present no difficulty
        whatever. The only consonants which call for special remark are the
        following: ш and ж, ч
        and џ, ћ and ђ.

        ш is a \textit{voiceless}\foot{The difference between a
        \textit{voiceless} and a \textit{voiced} consonant is that a
        \textit{voiceless} consonant is pronounced with breath from the mouth
        \textit{only}, while to pronounce a \textit{voiced} consonant a stream of
        breath from the chest is necessary.} consonant exactly like English \textit{sh}
        ; ж is the corresponding \textit{voiced} consonant pronounced
        like \textit{s} in the English word \textit{pleasure}, or like \textit{j} in
        the French word \textit{jour}.

        ч is a \textit{voiceless} consonant exactly like English
        \textit{ch} in \textit{chalk} ; џ is the corresponding
        \textit{voiced} consonant pronounced like \textit{j} in the English
        word \textit{John}.

        \begin{samepage} The only difficulty is with the two consonants ћ and
        ђ, though it is by no means insurmountable. To pronounce
        these two consonants the teeth must be brought close together and the
        lips slightly opened. The blade\foot{The blade is the part of the
        tongue immediately behind the point and including it.} of the tongue
        must cleave to the inside of the gums of the upper teeth and be
        slightly drawn \end{samepage}  back at the moment when the stream of breath comes out
        of the chest throught the mouth. The important point is that
        ћ is a \textit{voiceless} and ђ the corresponding \textit{voiced}
        consonant. Thus ћ and ђ correspond to ч and џ and are very similar to
        them in sound, only they are palatal consonants,\foot{ћ and ђ, besides
        being the result respectively of т+ј and д+ј (cf. p. 18), are also the
        result, in words of comparatively modern formation, of к+ј and к+е, г+ј,
        г+е, e.g. ћ\`{о}шак = \emph{corner} (from Turkish \emph{kiushk}, cf.
        \emph{kiosque, a pavilion}), Маћ\`{е}д\={о}нија=\emph{Macedonia} (к+е),
        М\`{а}ђ\={а}р (also M\`{а}џ\={а}р) = \emph{Magyar}, Ђ\trc{о}рђе = \emph{George},
        ђен\`{е}р\={а}л = \emph{general}, though there is now no к or г audible in
        these words.} which ч and џ are not.

        The consonant {х} before a consonant, as in
        хв\'{а}ла = \textit{thanks}, is pronounced like \textit{ch}
        in Scottish \textit{loch}, but before a vowel like an ordinary English
        \textit{h}, as in х\`{а}ртија = \textit{paper}.

        It is important to notice the difference between л and љ, and between н
        and њ ; љ and њ are the \textit{softened} or \textit{palatal} forms of
        л and н, just as ћ and ђ are the \textit{softened} or \textit{palatal}
        forms of т and д. Their pronunciation is perfectly easy and natural for
        English-speaking people except at the end of words, a position in which
        for that matter these letters in Serbo-Croatian seldom occur; in the
        middle of words they sound like \textit{l} and \emph{n} in the English
        words \emph{million} and \emph{new}. 



\vspace*{16pt}
\chapter{THE DIFFERENT KINDS OF SOUNDS}
\section{Vowels}

Besides the five normal vowels--а, е, и, о, у--р can also rank as a vowel when
it is (1) between to consonants, or (2) at the beginning of a word before a
consonant; in these cases it is strongly rolled as in Scotland, e.g.

1. С\textdoublegrave{р}бин=\emph{a Serb} (\emph{masc.}) ;
т\`{р}говац=\emph{merchant} ; чв\trc{р}ст=\emph{firm}.\foot{Even in words of
foreign origin, e.g. трпез\`{а}рија=\emph{dining-room}, from the Greek
\textgreek{τραπεζάριον}.}

2. \textdoublegrave{р}ђа=\emph{rust} ; \`{р}вати се=\emph{to wrestle} ;
\textdoublegrave{р}ђав=\emph{bad}.

р  very seldom occurs as a vowel-sound before or after a vowel; when it does it
is  indicated by two dots, e.g.
\newpage

г\"{р}\'{о}це (3 syllables) = \emph{throat} (diminutive).

за\`{\"{р}}ђати (4 syllables) = \emph{to become rusty}.

All vowels, including р, may be either short or long.

\smallskip
{\hfill \large \textbf{`Movable A'} \hfill}

Particular mention must be made of what is knows as the `movable a'. In Serbian
only the following four groups of consonants are possible at the end of words :
ст, шт, зд, жд; when a words would end in any other group than these, an a is
inserted in the \emph{nom, sing.}, but disappears in the other cases where the
word naturally ends in a vowel; but in the \emph{gen. plur.} the a reappears in
these words, a phenomenon caused by the fact that the invariable long
\emph{final} a of this case is of comparatively modern origin.  E.g.

\emph{Nom. sing.} к\`{о}нац=\emph{cotton, cotton, thread}

\emph{Gen. sing.} к\'{о}нца.

\emph{Gen. plur.} к\`{о}н\={а}ц\={а}

It is very frequent in the \emph{nom, sing. masc.} of adjectives, e.g.
ж\'{е}дан (\emph{masc})=\emph{thirsty}, but ж\'{е}дна (\textit{fem.}).

In the case of foreign words practice varies ; thus one finds both
ф\textdoublegrave{а}кт and ф\textdoublegrave{а}кат. 

\smallskip
{\hfill \large \textbf{Final л and о} \hfill}

Final л of a syllable, and especially of a word, very frequently becomes о.
In words which originally ended in -ол in the \emph{nom. sing.} the two о's
then combine into one long vowel, but the л reappears in the othercases, e.g.

в\trc{о} (\emph{m.})=\emph{ox}, gen. sing. в\'{о}ла. 

ст\trc{о} (\emph{m.})=\emph{Table}, gen.sing. ст\`{о}ла.

с\trc{о}=\emph{salt}, gen.sing. с\`{о}ли, the nom. sing. of which was originally
вол, стол, and сол.

In other cases the л appears as о after another vowel when
final, reappearing in other cases, e.g.

б\textdoublegrave{е}о=\emph{white} (nom. sing, masc), but б\'{е}ла
=\emph{white} (nom. sing, fem.), б\trc{е}ли=\emph{white} (nom. plur. masc).
This phenomenon occurs most frequently in the past participle of the verbs, e.g.

\`{и}мао=\emph{(he) had} (masc. sing.), but \`{и}мала=\emph{(she) had} (fem. sing.). 

\pagebreak

It may also occur in the middle of words when n. is at the end
of a syllable, e.g.

се\`{о}ба=\emph{migration} (originally селба).

вл\'{а}даоца, gen. sing. of вл\`{а}далац=\emph{ruler} (e,g, \emph{king}).

Cf. also Бе\`{о}град=\emph{Belgrade} (lit. \emph{the white city}, originally Бел-град).

\section{Consonants}

The consonants, according to the manner of their
articulation, fall into the two groups :

1. Voiced : б, в, г, д, ђ, ж, з, џ.

2. Voiceless : п, ф, к, т, ћ, ш, с, ч, ц, х.

\smallskip
{\hfill \textit{Rule of the assimilation of Consonants} \hfill}

When a voiced and a voiceless consonant come together,\linebreak
assimilation takes place, i.e. both must be either voiced \linebreak 
or voiceless : (1) a voiceless consonant becomes voiced before
a voiced consonant, and (2) vice versa, e.g.

\begin{itemize}
    \item[(1)] св\textdoublegrave{а}дба (\emph{f.})=\emph{wedding} is derived
        from сват+ба (св\textdoublegrave{а}т (\emph{m}.)\\
        \hspace*{3in}=\emph{wedding guest})\\
        \`{о}таџбина ({\it f.}) = \textit{fatherland}\quad ,,\qquad,, \quad
        отач+бина (\`{о}тац ({\it m.})\\
        \hspace*{3in}=\emph{father})  
    \item[(2)] с\textdoublegrave{р}пски ({\it adj.}) =
        \emph{Serbian}\qquad,,\qquad,, \quad 
        срб+ски (с\textdoublegrave{р}бин ({\it m.}))\\
        \hspace*{3in}=\emph{Serbian} ({\it m.})\\
        вр\'{а}пци ({\it nom. pl.}) =\emph{sparrows}\qquad,, \quad
        враб + ци\quad(вр\'{а}бац\\
        \hspace*{3in}({\it m.})= \emph{sparrow})
\end{itemize}

Exceptions: д remains before с and ш, e.g.

\qquad\qquad пр\'{е}дседник ({\it m.}) = \emph{president.}

\qquad\qquad одшкринути = \emph{to open slightly.}

в never changes into ф and does not change preceding voiceless consonants, e.g.

\qquad\qquad к\`{о}л\={е}вка ({\it f.}) (not колефка) = \textit{cradle}.

\qquad\qquad кл\'{е}тва ({\it f.}) (not кледва) = \textit{curse}.

\grammarnote{Most Important Phonetic Rules}

I. The gutterals к, г, х are `softened' when followed (1) by е and (2) by и, as follows :
\newpage

1. ({\it a.}) Before е, к changes into ч, г into ж, х into ш, in voc. sing. of masculine nouns, e.g.

Nom. sing., v\trc{у}к ({\it m.}), \emph{wolf}, voc. sing. в\trc{у}че.

\quad,, \quad ,, \qquad б\textroundcap{о}г \masc, \emph{god},\qquad,,\quad,,\quad б\tdg{о}же.

\quad,, \quad ,, \qquad д\tdg{у}х \masc, \emph{spirit},\quad,,\quad,,\quad д\tdg{у}ше.

({\it b}) In the 2nd and 3rd person singular of the aorist tense (cd. p. 187f.) e.g. р\`{е}ћи = \emph{to tell}, т\tdg{р}гнути = \emph{to pull}.

1st p. р\tdg{е}кох, \emph{I told}, \quad 2nd and 3rd p. р\tdg{е}че. 

\quad,,\quad т\tdg{р}гох, \emph{I pulled}, \quad,, \quad \qquad,,\qquad т\tdg{р}же.

({\it c}) In certain words derived from those ending in these consonants, e.g.

др\textroundcap{у}г \masc, \emph{companion} ; др\'{у}жити се, \emph{to keep company}.

к\`{о}нак \masc, \emph{a hostel} ; к\`{о}начити, \emph{to spend the night}.

с\textroundcap{у}х ({\it adj}), \emph{dry} ; с\'{у}шити, \emph{to dry} (transitive). 

2. Before и, г changes into з, к into ц, х into с, in the nom. dat. voc. inst.
loc. pl. of most nouns whose stems end in these consonants, e.g.

б\'{у}брег \masc, \emph{kidney}, nom. voc. pl. б\'{у}брези, dat. inst. loc. б\'{у}брезима. 

в\textroundcap{у}к \masc, \emph{wolf}, nom. voc. pl. в\textroundcap{у}ци, dat. inst. loc. в\textroundcap{у}цима.  

сир\'{о}мах \masc, \emph{poor man}, nom. voc. pl. сир\'{о}маси, dat. inst. loc. сир\'{о}масима.

II. If ц and з are followed by е or и, they become ч and ж, e.g.

з\textroundcap{е}ц \masc, \emph{hare} ({\it masc}), voc. sing. з\textroundcap{е}че, з\`{е}чица \femn, \emph{hare} ({\it fem.}).

кн\trc{е}з \masc, \emph{price}, \qquad,,\quad,,\quad кн\trc{е}же.

\`{о}тац \masc,\emph{father},\qquad,,\quad,,\quad \`{о}че.

III. In the case of verbs whose roots end in г, к, and х, these consonants
coalesce with т of the infinitive ending -ти and form ћ, cf. p. 102.


IV. The palatal consonant j, in such syllables as -ја-, -је-, -ји-, -ју-,
affects most of the non-palatal consonants if they immediately precede it. Such
consonants coalesce with j into one sound, as follows :\\ 
д+ј=ђ, e.g. мл\tdg{а}ђ\={и} ({\it adj.}) = \emph{younger}, derived from млад +
ји\footnote{-ји s the sign of the comparative.}\\ 
т+ј=ћ, e.g. љ\tdg{у}ћ\={и} ({\it adj.}) = \emph{more angry}\quad,,\qquad,,\quad љут\quad\hspace*{-2pt}+ ји

\newpage

\noindent з+ј=ж e.g. б\tdg{р}ж\={и} ({\it adj})= \emph{quicker}, derived from брз\quad+ ји\\
с+ј=ш, e.g. к\tdg{и}ша \femn = \emph{rain}\qquad\qquad,,\qquad ,,\quad кис\quad+ ја\\
ц+ј=ч, e.g. \tdg{у}жичанин \masc = \emph{a native}\\
\hspace*{0.42\linewidth} \emph{of \trc{У}жице},,\qquad ,,\quad ужиц + јанин\\
и+ј=њ, e.g. т\tdg{а}њ\={и} ({\it adj.}) = \emph{thinner} \quad ,,\qquad ,,\quad тан\quad + ји\\
л+ј=љ, e.g. вес\'{е}ље ({\it n.}) = \emph{joy} \quad\qquad,,\qquad ,,\quad весел + је\\
г+ј=ж, e.g. др\tdg{а}ж\={и} ({\it adj.}) = \emph{dearer} \quad,,\qquad ,,\quad драг\quad+ ји\\
к+ј=ч, e.g. ј\tdg{а}ч\={и} ({\it adj.}) = \emph{stronger} \quad ,,\qquad ,, \quad јак\quad + ји\\
х+ј=ш, e.g. т\tdg{и}ш\={и} ({\it adj.}) = \emph{quieter} \quad,,\qquad ,,\quad тих\quad + ји\\

Further, if such consonants are in their turn preceded by з or с, these become
respectively ж and ш, e.g.\\

гр\trc{о}зд \masc = \emph{bunch of grapes}, but гр\trc{о}жђе \neut \emph{grapes} (collective noun), from грозђе = грозд-је.

л\trc{и}ст \masc = \emph{leaf, sheet (of paper)}, but л\trc{и}шће \neut = \emph{leaves} (collective noun), from лисће = лист-је.

Б\tdg{о}сна \femn = \emph{Bosnia}, but Б\tdg{о}шн\={а}к \masc = \emph{a Bosnian} \masc, from Босњак = Босн-јак.\\

Further, when the syllables beginning with ј are immediately preceded by the consonants б, п, в, м, the letter л is inserted and coalesces with ј, forming the consonant љ, e.g.


\end{document}
